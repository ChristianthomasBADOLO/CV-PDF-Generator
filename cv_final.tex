\documentclass[11pt]{article}
\usepackage{graphicx} % Required for inserting images
\setlength{\parindent}{0pt}
\usepackage{hyperref}
\usepackage{enumitem}
\usepackage[utf8]{inputenc} 
\usepackage[T1]{fontenc}
\usepackage[brazil]{babel}
\usepackage{lipsum}
\usepackage[left=1.06cm,top=1.7cm,right=1.06cm,bottom=0.49cm]{geometry}
\usepackage{multicol}
\usepackage{colortbl}
\usepackage{tikz}
\usepackage{xcolor}

%by: Aline R. Antunes
\usetikzlibrary{decorations.pathmorphing, decorations.text}

\newcommand{\widehline}{~}
\newcommand{\header}[1]{\noindent
  \begin{tikzpicture}[baseline=0,trim left=0pt]
    \fill [black!5] (-0.1,0) rectangle (10.1,1.01);
    \node at (5,0.52) {\textbf{#1}};
  \end{tikzpicture}
    \vspace{0.5pt}
    \hrule height 1pt \vspace{0.5pt}}

\newcommand{\section}[1]{\noindent
  \begin{tikzpicture}[baseline=-1pt,trim left=0pt]
    \node at (0,-1.3) {\textbf{#1}};
  \end{tikzpicture}}

\begin{document}

\begin{tabular}{rl}
    \begin{tabular}{c}
        \Large\textbf{Christian Thomas BADOLLO}\\
        \texttt{Élève-ingénieur}
    \end{tabular} &
    \begin{tabular}{r}
        Bouskoura, Maroc  \\
        +212 700-127382 \\
        christian.badolo@centrale-casablanca.ma \\
        www.linkedin.com/in/christianthomasbadolo \\
    \end{tabular}
    \\\
    \multicolumn{2}{l}{\header{PROFIL}}
\end{tabular}

Je suis élève-ingénieur généraliste (BAC+4) de la prestigieuse École Centrale Casablanca. Je suis intéressé par les solutions informatiques axées sur le développement durable et la RSE, avec un intérêt particulier pour l'analyse et le traitement des données. Spontané, rigoureux et curieux, je recherche activement un stage d'assistant ingénieur de 12 semaines minimum dans ce sens.

\multicolumn{2}{l}{\header{EXPÉRIENCES}}

\begin{multicols*}{2}
\begin{itemize}[noitemsep, topsep=0pt, partopsep=0pt, parsep=0pt]
    \item Responsable événementiel - ENACTUS École Centrale Casablanca (ECC), avril 2023 - maintenant
        \begin{itemize}[noitemsep, topsep=0pt, partopsep=0pt, parsep=0pt]
            \item Organisation d'ateliers
        \end{itemize}
    \item Stage en Data Analytics - CodSoft, janvier 2024 - février 2024
        \begin{itemize}[noitemsep, topsep=0pt, partopsep=0pt, parsep=0pt]
            \item Nettoyage de données
            \item Calcul des statistiques sommaires
            \item Visualisation avec des histogrammes
            \item Création de tableaux croisés dynamiques
            \item Développement de tableaux de bord avec PowerBI
        \end{itemize}
    \item Projet « GreenVoice » - ECC, septembre 2023 - janvier 2024
        \begin{itemize}[noitemsep, topsep=0pt, partopsep=0pt, parsep=0pt]
            \item Création d'une solution axée sur la RSE et le développement durable pour les grandes distributions (Marjane)
            \item Utilisation des LLM pour le développement de voicebot
        \end{itemize}
    \item Article scientifique « Predictive maintenance with optimal transport » - ECC, janvier 2024
        \begin{itemize}[noitemsep, topsep=0pt, partopsep=0pt, parsep=0pt]
            \item Conception d'un modèle de prévision de séries temporelles avec TensorFlow
            \item Utilisation de la méthode du transport optimal pour la détection d'anomalies.
        \end{itemize}
    \item Stage opérateur - Smartprof, juillet - août 2023
        \begin{itemize}[noitemsep, topsep=0pt, partopsep=0pt, parsep=0pt]
            \item Web scraping
            \item Conception et mise au point d'un moteur de recherche
            \item SEO
        \end{itemize}
    \item Projet « SmartRamp » - ECC, septembre 2022 - juin 2023
        \begin{itemize}[noitemsep, topsep=0pt, partopsep=0pt, parsep=0pt]
            \item Facilitateur de l'équipe
            \item Responsable technique : conception de site web dynamique
        \end{itemize}
\end{itemize}
\end{multicols*}

\multicolumn{2}{l}{\header{EDUCATION}}

\begin{tabular}{l}\small
- Actuellement - École Centrale Casablanca (ECC) \\
2\textsuperscript{ème} année cycle ingénieur : Industrie 4.0 \\
- Communication et réseaux industriels, internet des objets (IoT) \\
- Transformation digitale : démarches et mise en œuvre \\
- New Business Models in Digital Economy \\
- IA et applications en industrie 4.0 \\\  
\\\
2\textsuperscript{ème} année cycle ingénieur : Modélisation et Aide à la Décision \\
- Développement durable - RSE \\
- Recherche opérationnelle / Optimisation / Simulation \\
- Contrôle qualité \\
\end{tabular}

\multicolumn{2}{l}{\header{LANGUES}}

- Français : Niveau C1, DALPH
- Anglais : Niveau B2+, TOEIC

\multicolumn{2}{l}{\header{CENTRES D'INTÉRÊT}}

- Basketball
- Musique
- Géopolitique
- Culture japonaise

\multicolumn{2}{l}{\header{HUMAN SKILLS}}

- Empathique
- Réceptif
- Rigoureux

\multicolumn{2}{l}{\header{HARD SKILLS}}

- Machine learning
- Python (pandas, tensorflow, scikit-learn)
- Automatisation et intégration continues (Docker, FastApi, Grafana \& Prometheus) \\\  \\\
- Développement web
- Python (Django), Javascript (React), HTML, php, SQL \\\  \\\
- Suite Office
- Microsoft Word, Excel (VBA), Powerpoint, PowerBI \\\  \\\
- Gestion de projets et collaboration
- Github, Trello

\multicolumn{2}{l}{\header{CONTACTS}}

- LinkedIn : \href{https://www.linkedin.com/in/christianthomasbadolo}{https://www.linkedin.com/in/christianthomasbadolo}
- Portfolio : \href{https://christianthomasbadolo.me/}{https://christianthomasbadolo.me/}

\end{document}